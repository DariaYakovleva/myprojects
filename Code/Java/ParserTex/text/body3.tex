
\thispagestyle{empty}
\begin{center}
{\Large\bfseries Министерство образования и науки Российской Федерации}\\
Федеральное государственное автономное образовательное учреждение\\
высшего образования\\
{\bfseries «Санкт-Петербургский национальный исследовательский университет\\
информационных технологий, механики и оптики»}

\vspace{1cm}

\begin{flushright}
УТВЕРЖДАЮ\\
Проректор по УМР\\
~\\
\_\_\_\_\_\_\_\_\_\_\_\_\_\_~Шехонин А.~А.\\
``\_\_\_\_''~\_\_\_\_\_\_\_\_\_\_\_~20\_\_~г.\\
м.п.
\end{flushright}

\vspace{1cm}

{\Large\bfseries РАБОЧАЯ ПРОГРАММА ДИСЦИПЛИНЫ}\\
{\Large\bfseries Б.3.1.7 «Функциональное программирование»}
{\hrule height 2pt}

\vspace{1cm}

\newlength{\pretext}
\newlength{\remaining}
\newcommand{\titleunderline}[2]{%
\setlength{\pretext}{\widthof{\textsc{#1~}}}%
\setlength{\remaining}{\textwidth-\pretext}%
\noindent\textsc{#1}~\underline{\makebox[\remaining]{\textsc{#2}}}\par}

\newcommand{\titleunderlineleft}[2]{%
\setlength{\pretext}{\widthof{\textsc{#1~}}}%
\setlength{\remaining}{\textwidth-\pretext-\widthof{\textsc{#2}}}%
\noindent\textsc{#1}~\underline{\textsc{#2}\hspace*{\remaining}}\par}

\newcommand{\titleunderlinecont}[1]{%
\setlength{\remaining}{\textwidth-\pretext-\widthof{\textsc{#1}}}%
\noindent\hspace*{\pretext}\underline{\textsc{#1}\hspace*{\remaining}}\par}

\begin{flushleft}
\titleunderline{\textbf{Направление подготовки}}{010302 «Прикладная математика и информатика»}\vspace{12pt}
\titleunderline{\textbf{Квалификация (степень) выпускника}}{бакалавр}\vspace{12pt}
\titleunderlineleft{\textbf{Профиль подготовки бакалавра}}{Математические модели и алгоритмы в}
\titleunderlinecont{разработке программного обеспечения}
\titleunderline{\textbf{Форма обучения}}{очная}\vspace{12pt}
\titleunderline{\textbf{Выпускающая кафедра}}{Компьютерных Технологий}
\end{flushleft}

\vspace{2cm}

\titleunderline{Кафедра-разработчик рабочей программы}{Компьютерных Технологий}

\vspace{1cm}

\begin{tabular}{!{\VRule}c!{\VRule}c!{\VRule}c!{\VRule}c!{\VRule}c!{\VRule}c!{\VRule}c!{\VRule}}\HLine
\bfseries Семестр &
\bfseries \pb{Трудо-\\емкость,\\ час.} &
\bfseries \pb{Лек-\\ций,\\ час.} &
\bfseries \pb{Практич.\\ занятий,\\ час.} &
\bfseries \pb{Лаборат.\\ работ,\\ час.} &
\bfseries \pb{СРС,\\ час.} &
\bfseries \pb{Форма промежуточного\\ контроля\\ (экзамен/зачет)}\\\HLine
7 & 136 & 34 & 0 & 34 & 68 & Экзамен\\\HLine

Итого: & 136 & 34 & 0 & 34 & 68 & \\\HLine
\end{tabular}

\vfill

Санкт-Петербург,\\
2014
\end{center}

\newpage
\section*{РАБОЧАЯ ПРОГРАММА ДИСЦИПЛИНЫ}

\paragraph{Разделы рабочей программы}
\begin{enumerate}
\item Цели освоения дисциплины
\item Место дисциплины в структуре ООП ВПО
\item Структура и содержание дисциплины
\item Формы контроля освоения дисциплины
\item Учебно-методическое и информационное обеспечение дисциплины
\item Материально-техническое обеспечение дисциплины
\end{enumerate}

\paragraph{Приложения к рабочей программе дисциплины}
{
\setenumerate[1]{label={Приложение \arabic{enumi}.}, fullwidth, itemindent=\parindent, listparindent=\parindent} 
\begin{enumerate}
\item Аннотация рабочей программы
\item Технологии и формы преподавания 
\item Технологии и формы обучения 
\item Оценочные средства и методики их применения
\item Таблица планирования результатов обучения
\end{enumerate}
}

\vspace{1cm}

Программа составлена в соответствии с требованиями ОС НИУ ИТМО по направлению подготовки: 010302 «Прикладная математика и информатика».

\vspace{1cm}

{\parindent0pt
Программу составили:\\
кафедра КТ\\ \_\_\_\_\_\_\_\_\_\_\_\_\_\_\_\_\_\_\_\_\_ Корнеев Г. А., кандидат технических наук, доцент\\ \_\_\_\_\_\_\_\_\_\_\_\_\_\_\_\_\_\_\_\_\_ Маврин П. Ю., старший преподаватель

\vspace{1cm}

Зав. кафедрой:\\
\_\_\_\_\_\_\_\_\_\_\_\_\_\_\_\_\_\_\_\_\_ Парфенов В. Г., доктор технических наук, профессор

\vspace{1cm}

Эксперт(ы):\\ \_\_\_\_\_\_\_\_\_\_\_\_\_\_\_\_\_\_\_\_\_ Шопырин Д. Г., генеральный директор ООО "ВСВН Лаб"

\vfill

Программа одобрена на заседании УМК факультета Информационных Технологий и Программирования.

\vspace{1cm}

Председатель УМК ИТиП:\\
\_\_\_\_\_\_\_\_\_\_\_\_\_\_\_\_\_\_\_\_\_ Харченко Т. В.
}

\newpage
\section{ЦЕЛИ ОСВОЕНИЯ ДИСЦИПЛИНЫ}
Целью освоения дисциплины является достижение следующих результатов образования (РО):\\

{\parindent0pt
\textbf{знания}:
\begin{itemize}
\item на уровне представлений:
\begin{itemize}
\item представление о принципах построения программ высокого качества на функциональных языках,\item представление о понятии функций высших порядков,\item лямбда-исчисление, интерпретация и компиляция функциональных програм,
\end{itemize}
\item на уровне воспроизведения:

\item на уровне понимания:
\begin{itemize}
\item понимание общих правил построения хорошего программного кода,
\end{itemize}
\end{itemize}
\textbf{умения}:\\
\begin{itemize}
\item теоретические:

\item практические:

\end{itemize}
\textbf{навыки}:
\begin{itemize}
\item навык разработки программ среднего размера, находить и устранять их возможные уязвимости,\item навык написания программ на языке программирования Haskell,\item навык проектирования программ на функциональном языке программирования,\item навык проведения вычислений в лямбда-исчислении,\item навык использования библиотек функциональных языков программирования.
\end{itemize}

Перечисленные РО являются основой для формирования следующих компетенций:
\begin{itemize}
\item общекультурных: \begin{itemize}
\item ОК.14 — способностью использовать в научной и познавательной деятельности, а также в социальной сфере профессиональные навыки работы с информационными и компьютерными технологиями,\item ОК.15 — способностью работы с информацией из различных источников, включая сетевые ресурсы сети Интернет, для решения профессиональных и социальных задач,\item ОК.16 — способностью к интеллектуальному, культурному, нравственному, физическому и профессиональному саморазвитию, стремление к повышению своей квалификации и мастерства,
\end{itemize}\item профессиональных: \begin{itemize}
\item ПК.3 — способностью понимать и применять в исследовательской и прикладной деятельности современный математический аппарат,\item ПК.4 — способностью в составе научно-исследовательского и производственного коллектива решать задачи профессиональной деятельности,\item ПК.5 — способностью критически переосмысливать накопленный опыт, изменять при необходимости вид и характер своей профессиональной деятельности,\item ПК.7 — способностью собирать, обрабатывать и интерпретировать данные современных научных исследований, необходимые для формирования выводов по соответствующим научным, профессиональным, социальным и этическим проблемам,
\end{itemize}
\end{itemize}
}

\newpage
\section{МЕСТО ДИСЦИПЛИНЫ В СТРУКТУРЕ ООП ВПО}
Дисциплина «Функциональное программирование» является частью базовой части профессионального цикла дисциплин.
Необходимыми условиями для освоения дисциплины являются:
\begin{itemize}
\item знания:
\begin{itemize}
\item знание базовых приемов, используемых при проектировании алгоритмов и структур данных,\item воспроизведение базовых алгоритмов, основанных на использовании основных структур данных,\item понимание базовых структур данных и операций над ними,\item технический английский,
\end{itemize}
\item умения:
\begin{itemize}
\item особенности сетевого взаимодействия операционных систем,
\end{itemize}
\item навыки:
\begin{itemize}
\item алгоритмическое мышление.
\end{itemize}
\end{itemize}


Содержание дисциплины является логическим продолжением содержания дисциплин: \begin{itemize}
\item Б.2.1.1 «Математический анализ»\item Б.2.1.2 «Алгебра и геометрия»\item Б.2.1.3 «Физика»\item Б.2.2.1.4 «Математическая физика»\item Б.2.2.1.5 «Функциональный анализ»\item Б.2.2.1.6 «Концепции современного естествознания»\item Б.2.2.1.7 «Численные методы»\item Б.3.1.1 «Безопасность жизнедеятельности»\item Б.3.1.10 «Языки программирования»\item Б.3.1.11 «Операционные системы»\item Б.3.1.2 «Дискретная математика»\item Б.3.1.3 «Алгоритмы и структуры данных»\item Б.3.1.4 «Теория формальных языков»\item Б.3.1.5 «Методы трансляции»\item Б.3.1.6 «Теория вероятностей и математическая статистика»\item Б.3.1.8 «Введение в программирование и ЭВМ»\item Б.3.1.9 «Технологии программирования»\item Б.3.2.1.1 «Автоматное программирование»\item Б.3.2.1.2 «Вычислительная геометрия»\item Б.3.2.1.3 «Параллельное программирование»\item Б.3.2.1.4 «Теория вычислительной сложности»\item Б.3.2.2.1 «Парадигмы программирования»\item Б.3.2.2.1 «Язык программирования Java»\item Б.3.2.2.2 «Алгоритмы в математике»\item Б.3.2.2.2 «Специальный семинар»\item Б.3.2.2.5 «Практикум на ЭВМ»\item Б.3.2.2.5 «Специальный семинар»\item Б.5.1 «Производственная практика»
\end{itemize} и служит основой для освоения дисциплин: \begin{itemize}
\item Б.2.2.1.8 «Теория игр и исследования операций»\item Б.3.1.13 «Методы оптимизации»\item Б.5.2 «Преддипломная практика»
\end{itemize}

\newpage
В таблице приведены предшествующие и последующие дисциплины, направленные на формирование компетенций, заявленных в разделе «Цели освоения дисциплины»:

\begin{longtable}{|c|p{0.15\textwidth}|p{0.35\textwidth}|p{0.35\textwidth}|}\hline
№ п/п &
\multicolumn{1}{c|}{\pb{Наименование\\компетенции}} &
\multicolumn{1}{c|}{Предшествующие дисциплины} &
\multicolumn{1}{c|}{\pb{Последующие дисциплины\\(группы дисциплин)}}\\\hline
\multicolumn{4}{|l|}{\textit{Общекультурные компетенции}}\\\hline 1 & ОК.14 & Б.2.1.1 «Математический анализ», Б.2.1.2 «Алгебра и геометрия», Б.2.1.3 «Физика», Б.2.2.1.4 «Математическая физика», Б.2.2.1.5 «Функциональный анализ», Б.2.2.1.6 «Концепции современного естествознания», Б.2.2.1.7 «Численные методы», Б.3.1.1 «Безопасность жизнедеятельности», Б.3.1.10 «Языки программирования», Б.3.1.11 «Операционные системы», Б.3.1.2 «Дискретная математика», Б.3.1.3 «Алгоритмы и структуры данных», Б.3.1.4 «Теория формальных языков», Б.3.1.5 «Методы трансляции», Б.3.1.6 «Теория вероятностей и математическая статистика», Б.3.1.8 «Введение в программирование и ЭВМ», Б.3.1.9 «Технологии программирования», Б.3.2.1.1 «Автоматное программирование», Б.3.2.1.2 «Вычислительная геометрия», Б.3.2.1.3 «Параллельное программирование», Б.3.2.1.4 «Теория вычислительной сложности», Б.3.2.2.1 «Парадигмы программирования», Б.3.2.2.1 «Язык программирования Java», Б.3.2.2.2 «Алгоритмы в математике», Б.3.2.2.2 «Специальный семинар», Б.3.2.2.5 «Практикум на ЭВМ», Б.3.2.2.5 «Специальный семинар», Б.5.1 «Производственная практика» & Б.2.2.1.8 «Теория игр и исследования операций», Б.3.1.13 «Методы оптимизации», Б.5.2 «Преддипломная практика»\\\hline
2 & ОК.15 & Б.2.1.1 «Математический анализ», Б.2.1.2 «Алгебра и геометрия», Б.2.1.3 «Физика», Б.2.2.1.4 «Математическая физика», Б.2.2.1.5 «Функциональный анализ», Б.2.2.1.6 «Концепции современного естествознания», Б.2.2.1.7 «Численные методы», Б.3.1.1 «Безопасность жизнедеятельности», Б.3.1.10 «Языки программирования», Б.3.1.11 «Операционные системы», Б.3.1.2 «Дискретная математика», Б.3.1.3 «Алгоритмы и структуры данных», Б.3.1.4 «Теория формальных языков», Б.3.1.5 «Методы трансляции», Б.3.1.6 «Теория вероятностей и математическая статистика», Б.3.1.8 «Введение в программирование и ЭВМ», Б.3.1.9 «Технологии программирования», Б.3.2.1.1 «Автоматное программирование», Б.3.2.1.2 «Вычислительная геометрия», Б.3.2.1.3 «Параллельное программирование», Б.3.2.1.4 «Теория вычислительной сложности», Б.3.2.2.1 «Парадигмы программирования», Б.3.2.2.1 «Язык программирования Java», Б.3.2.2.2 «Алгоритмы в математике», Б.3.2.2.2 «Специальный семинар», Б.3.2.2.5 «Практикум на ЭВМ», Б.3.2.2.5 «Специальный семинар», Б.5.1 «Производственная практика» & Б.2.2.1.8 «Теория игр и исследования операций», Б.3.1.13 «Методы оптимизации», Б.5.2 «Преддипломная практика»\\\hline
3 & ОК.16 & Б.2.1.1 «Математический анализ», Б.2.1.2 «Алгебра и геометрия», Б.2.1.3 «Физика», Б.2.2.1.4 «Математическая физика», Б.2.2.1.5 «Функциональный анализ», Б.2.2.1.6 «Концепции современного естествознания», Б.2.2.1.7 «Численные методы», Б.3.1.1 «Безопасность жизнедеятельности», Б.3.1.10 «Языки программирования», Б.3.1.11 «Операционные системы», Б.3.1.2 «Дискретная математика», Б.3.1.3 «Алгоритмы и структуры данных», Б.3.1.4 «Теория формальных языков», Б.3.1.5 «Методы трансляции», Б.3.1.6 «Теория вероятностей и математическая статистика», Б.3.1.8 «Введение в программирование и ЭВМ», Б.3.1.9 «Технологии программирования», Б.3.2.1.1 «Автоматное программирование», Б.3.2.1.2 «Вычислительная геометрия», Б.3.2.1.3 «Параллельное программирование», Б.3.2.1.4 «Теория вычислительной сложности», Б.3.2.2.1 «Парадигмы программирования», Б.3.2.2.1 «Язык программирования Java», Б.3.2.2.2 «Алгоритмы в математике», Б.3.2.2.2 «Специальный семинар», Б.3.2.2.5 «Практикум на ЭВМ», Б.3.2.2.5 «Специальный семинар», Б.5.1 «Производственная практика» & Б.2.2.1.8 «Теория игр и исследования операций», Б.3.1.13 «Методы оптимизации», Б.5.2 «Преддипломная практика»\\\hline
\multicolumn{4}{|l|}{\textit{Профессиональные компетенции}}\\\hline 1 & ПК.3 & Б.2.1.1 «Математический анализ», Б.2.1.2 «Алгебра и геометрия», Б.2.1.3 «Физика», Б.2.2.1.4 «Математическая физика», Б.2.2.1.5 «Функциональный анализ», Б.2.2.1.6 «Концепции современного естествознания», Б.2.2.1.7 «Численные методы», Б.3.1.1 «Безопасность жизнедеятельности», Б.3.1.10 «Языки программирования», Б.3.1.11 «Операционные системы», Б.3.1.2 «Дискретная математика», Б.3.1.3 «Алгоритмы и структуры данных», Б.3.1.4 «Теория формальных языков», Б.3.1.5 «Методы трансляции», Б.3.1.6 «Теория вероятностей и математическая статистика», Б.3.1.8 «Введение в программирование и ЭВМ», Б.3.1.9 «Технологии программирования», Б.3.2.1.1 «Автоматное программирование», Б.3.2.1.2 «Вычислительная геометрия», Б.3.2.1.3 «Параллельное программирование», Б.3.2.1.4 «Теория вычислительной сложности», Б.3.2.2.1 «Парадигмы программирования», Б.3.2.2.1 «Язык программирования Java», Б.3.2.2.2 «Алгоритмы в математике», Б.3.2.2.2 «Специальный семинар», Б.3.2.2.5 «Практикум на ЭВМ», Б.3.2.2.5 «Специальный семинар», Б.5.1 «Производственная практика» & Б.2.2.1.8 «Теория игр и исследования операций», Б.3.1.13 «Методы оптимизации», Б.5.2 «Преддипломная практика»\\\hline
2 & ПК.4 & Б.2.1.1 «Математический анализ», Б.2.1.2 «Алгебра и геометрия», Б.2.1.3 «Физика», Б.2.2.1.4 «Математическая физика», Б.2.2.1.5 «Функциональный анализ», Б.2.2.1.6 «Концепции современного естествознания», Б.2.2.1.7 «Численные методы», Б.3.1.1 «Безопасность жизнедеятельности», Б.3.1.10 «Языки программирования», Б.3.1.11 «Операционные системы», Б.3.1.2 «Дискретная математика», Б.3.1.3 «Алгоритмы и структуры данных», Б.3.1.4 «Теория формальных языков», Б.3.1.5 «Методы трансляции», Б.3.1.6 «Теория вероятностей и математическая статистика», Б.3.1.8 «Введение в программирование и ЭВМ», Б.3.1.9 «Технологии программирования», Б.3.2.1.1 «Автоматное программирование», Б.3.2.1.2 «Вычислительная геометрия», Б.3.2.1.3 «Параллельное программирование», Б.3.2.1.4 «Теория вычислительной сложности», Б.3.2.2.1 «Парадигмы программирования», Б.3.2.2.1 «Язык программирования Java», Б.3.2.2.2 «Алгоритмы в математике», Б.3.2.2.2 «Специальный семинар», Б.3.2.2.5 «Практикум на ЭВМ», Б.3.2.2.5 «Специальный семинар», Б.5.1 «Производственная практика» & Б.2.2.1.8 «Теория игр и исследования операций», Б.3.1.13 «Методы оптимизации», Б.5.2 «Преддипломная практика»\\\hline
3 & ПК.5 & Б.2.1.1 «Математический анализ», Б.2.1.2 «Алгебра и геометрия», Б.2.1.3 «Физика», Б.2.2.1.4 «Математическая физика», Б.2.2.1.5 «Функциональный анализ», Б.2.2.1.6 «Концепции современного естествознания», Б.2.2.1.7 «Численные методы», Б.3.1.1 «Безопасность жизнедеятельности», Б.3.1.10 «Языки программирования», Б.3.1.11 «Операционные системы», Б.3.1.2 «Дискретная математика», Б.3.1.3 «Алгоритмы и структуры данных», Б.3.1.4 «Теория формальных языков», Б.3.1.5 «Методы трансляции», Б.3.1.6 «Теория вероятностей и математическая статистика», Б.3.1.8 «Введение в программирование и ЭВМ», Б.3.1.9 «Технологии программирования», Б.3.2.1.1 «Автоматное программирование», Б.3.2.1.2 «Вычислительная геометрия», Б.3.2.1.3 «Параллельное программирование», Б.3.2.1.4 «Теория вычислительной сложности», Б.3.2.2.1 «Парадигмы программирования», Б.3.2.2.1 «Язык программирования Java», Б.3.2.2.2 «Алгоритмы в математике», Б.3.2.2.2 «Специальный семинар», Б.3.2.2.5 «Практикум на ЭВМ», Б.3.2.2.5 «Специальный семинар», Б.5.1 «Производственная практика» & Б.2.2.1.8 «Теория игр и исследования операций», Б.3.1.13 «Методы оптимизации», Б.5.2 «Преддипломная практика»\\\hline
4 & ПК.7 & Б.2.1.1 «Математический анализ», Б.2.1.2 «Алгебра и геометрия», Б.2.1.3 «Физика», Б.2.2.1.4 «Математическая физика», Б.2.2.1.5 «Функциональный анализ», Б.2.2.1.6 «Концепции современного естествознания», Б.2.2.1.7 «Численные методы», Б.3.1.1 «Безопасность жизнедеятельности», Б.3.1.10 «Языки программирования», Б.3.1.11 «Операционные системы», Б.3.1.2 «Дискретная математика», Б.3.1.3 «Алгоритмы и структуры данных», Б.3.1.4 «Теория формальных языков», Б.3.1.5 «Методы трансляции», Б.3.1.6 «Теория вероятностей и математическая статистика», Б.3.1.8 «Введение в программирование и ЭВМ», Б.3.1.9 «Технологии программирования», Б.3.2.1.1 «Автоматное программирование», Б.3.2.1.2 «Вычислительная геометрия», Б.3.2.1.3 «Параллельное программирование», Б.3.2.1.4 «Теория вычислительной сложности», Б.3.2.2.1 «Парадигмы программирования», Б.3.2.2.1 «Язык программирования Java», Б.3.2.2.2 «Алгоритмы в математике», Б.3.2.2.2 «Специальный семинар», Б.3.2.2.5 «Практикум на ЭВМ», Б.3.2.2.5 «Специальный семинар», Б.5.1 «Производственная практика» & Б.2.2.1.8 «Теория игр и исследования операций», Б.3.1.13 «Методы оптимизации», Б.5.2 «Преддипломная практика»\\\hline

\end{longtable}

\newpage
\section{СТРУКТУРА И СОДЕРЖАНИЕ ДИСЦИПЛИНЫ}
Общая трудоемкость дисциплины составляет 4 зачетных единиц, 136 часов.

\begin{center}
\begin{longtable}{|c|c|p{0.25\textwidth}|p{1.4cm}|p{1.4cm}|p{1.4cm}|p{1.4cm}|p{1.4cm}|}\hline
\multicolumn{1}{|c|}{\multirow{2}{*}{\pb{\bfseries~\\№\\модуля}}} &
\multicolumn{1}{c|}{\multirow{2}{*}{\pb{\bfseries~\\№\\раздела}}} &
\multicolumn{1}{c|}{\multirow{2}{*}{\pb{\bfseries~\\Наименование\\раздела\\дисциплины}}} &
\multicolumn{5}{c|}{\pb{\bfseries{}Виды учебной нагрузки и их\\ трудоемкость, часы}}\\\cline{4-8}
& & &
\multicolumn{1}{c|}{\bfseries\begin{sideways}Лекции\end{sideways}} &
\multicolumn{1}{c|}{\bfseries\begin{sideways}\pb{Практические~~\\занятия}\end{sideways}} &
\multicolumn{1}{c|}{\bfseries\begin{sideways}\pb{Лабораторные~~\\работы}\end{sideways}} &
\multicolumn{1}{c|}{\bfseries\begin{sideways}СРС\end{sideways}} &
\multicolumn{1}{c|}{\bfseries\begin{sideways}Всего часов\end{sideways}}\\\hline
13 & 1 & Понятие о функциональном программировании & 18 & 0 & 18 & 36 & 72\\\hline
14 & 2 & Интерпретация и компиляция функциональных программ & 16 & 0 & 16 & 32 & 64\\\hline

\end{longtable}
\end{center}

\subsection{Содержание (дидактика) дисциплины}
% XXX Тут было решено положить болт на дидактические еденицы. Спасибо за понимание.

\begin{description}
\item[Раздел 1.] «Понятие о функциональном программировании».\item[Раздел 2.] «Интерпретация и компиляция функциональных программ».
\end{description}

\subsection{Лекции}

\begin{center}
\begin{longtable}{|c|c|c|p{0.6\textwidth}|}\hline
\multicolumn{1}{|c|}{\pb{\bfseries~\\№\\п/п\\~}} &
\multicolumn{1}{c|}{\pb{\bfseries №\\раздела}} &
\multicolumn{1}{c|}{\pb{\bfseries Объем,\\часов}} &
\multicolumn{1}{c|}{\bfseries Тема лекции} \\\hline
1 & 1 & 4 & Функциональное программирование\\\hline
2 & 1 & 4 & Введение в язык Haskell\\\hline
3 & 1 & 5 & Функции высших порядков\\\hline
4 & 1 & 5 & Лямбда-исчисление\\\hline
5 & 2 & 8 & Представление функциональных программ\\\hline
6 & 2 & 8 & Интерпретация\\\hline

\multicolumn{2}{|c|}{Итого:} & 34 & \\\hline
\end{longtable}
\end{center}


\subsection{Практические занятия}
Не предусмотрены.

\subsection{Лабораторные работы}

\begin{center}
\begin{longtable}{|c|c|c|p{0.4\textwidth}|p{0.2\textwidth}|}\hline
\multicolumn{1}{|c|}{\pb{\bfseries~\\№\\п/п\\~}} &
\multicolumn{1}{c|}{\pb{\bfseries №\\раздела}} &
\multicolumn{1}{c|}{\pb{\bfseries Трудоемкость,\\часов}} &
\multicolumn{1}{c|}{\pb{\bfseries Наименование лабораторной\\работы}} &
\multicolumn{1}{c|}{\pb{\bfseries Наименование\\лаборатории}} \\\hline
1 & 1 & 4 & Функциональное программирование: проба пера & Компьютерный класс\\\hline
2 & 1 & 4 & Рекурсия & Компьютерный класс\\\hline
3 & 1 & 5 & Работа со строками & Компьютерный класс\\\hline
4 & 1 & 5 & Дерево & Компьютерный класс\\\hline
5 & 2 & 8 & Граф & Компьютерный класс\\\hline
6 & 2 & 8 & Лямбда-исчисление & Компьютерный класс\\\hline

\multicolumn{2}{|c|}{Итого:} & 34 & & \\\hline
\end{longtable}
\end{center}


\subsection{Самостоятельная работа студента}

\begin{center}
\begin{longtable}{|c|c|c|p{0.6\textwidth}|}\hline
\multicolumn{1}{|c|}{\pb{\bfseries~\\№\\п/п\\~}} &
\multicolumn{1}{c|}{\pb{\bfseries №\\раздела}} &
\multicolumn{1}{c|}{\pb{\bfseries Трудоемкость,\\часов}} &
\multicolumn{1}{c|}{\pb{\bfseries Вид СРС}} \\\hline
1 & 1 & 18 & Подготовка к лекциям\\\hline
2 & 1 & 18 & Выполнение лабораторных работ\\\hline
3 & 2 & 16 & Подготовка к лекциям\\\hline
4 & 2 & 16 & Выполнение лабораторных работ\\\hline

\multicolumn{2}{|c|}{Итого:} & 68 & \\\hline
\end{longtable}
\end{center}


\subsection{Домашние задания, типовые расчеты и т.п.}
Не предусмотрены.

\subsection{Рефераты}
Не предусмотрены.

\subsection{Курсовые работы по дисциплине}
Не предусмотрены.

\newpage
\section{ФОРМЫ КОНТРОЛЯ ОСВОЕНИЯ ДИСЦИПЛИНЫ}

Текущий контроль успеваемости по дисциплине и промежуточная аттестация студентов по результатам семестра осуществляются в соответствии с положением о проведении текущего контроля успеваемости и промежуточной аттестации студентов НИУ ИТМО.

\textbf{Текущая аттестация} студентов производится лектором и преподавателем (преподавателями), ведущими лабораторные работы и практические занятия по дисциплине в следующих формах:
\begin{itemize}


\item выполнение лабораторных работ;
\item защита лабораторных работ;
\item отдельно оцениваются личностные качества студента.
\end{itemize}

\textbf{Рубежная аттестация} студентов производится по окончании модуля в следующих формах:
\begin{itemize}
\item защита лабораторных работ.
\end{itemize}

\textbf{Промежуточный контроль} по результатам семестра по дисциплине проходит:
\begin{itemize}
\item в форме устного экзамена в семестре №7.
\end{itemize}

Фонды оценочных средств, включающие типовые задания, контрольные работы, тесты и методы контроля, позволяющие оценить РО по данной дисциплине, включены в состав УМК дисциплины и перечислены в Приложении 4.
Критерии оценивания, перечень контрольных точек и таблица планирования результатов обучения приведены в Приложениях 4 и 5 к Рабочей программе.

\newpage
\section{УЧЕБНО-МЕТОДИЧЕСКОЕ И ИНФОРМАЦИОННОЕ ОБЕСПЕЧЕНИЕ ДИСЦИПЛИНЫ}

\setenumerate[1]{label={\alph{enumi})}} 
\setenumerate[2]{label={[\arabic{enumii}]},ref={\arabic{enumii}}} 
\begin{enumerate}
\item основная литература:
\begin{enumerate}
\item \label{funk} Кубенский А. А. Функциональное программирование: учебно-методическое пособие. — СПб.: СПбГУ ИТМО, 2010. \item \label{shalyto} Поликарпова Н. И., Шалыто А. А. Автоматное программирование: учебно-методическое пособие. — СПб.: СПбГУ ИТМО, 2008. \item \label{wirth} Вирт Н. Алгоритмы и структуры данных. — СПб.: Невский Диалект, 2008. \item \label{kormen} Кормен T., Лейзерсон Ч., Ривест Р., Штайн К. Алгоритмы: построение и анализ. — М.: Вильямс, 2012.
\end{enumerate}
 \item дополнительная литература:

 \item программное обеспечение, Интернет-ресурсы, электронные библиотечные системы:
\begin{enumerate}[resume]
\item \label{haskell} Haskell online\\ \url{https://tryhaskell.org/}
\end{enumerate}
\end{enumerate}

\newpage
\section{МАТЕРИАЛЬНО-ТЕХНИЧЕСКОЕ ОБЕСПЕЧЕНИЕ ДИСЦИПЛИНЫ}

\setenumerate[1]{label={\arabic{enumi}.}} 
\setenumerate[2]{label={\alph{enumii}.}} 

\begin{enumerate}
\item Лекционные занятия:
\begin{enumerate}
\item аудитория, оснащенная маркерной доской, \item комплект электронных презетраций/слайдов, \item презентационная техника (проектор, экран, компьютер/ноутбук).
\end{enumerate}
\item Лабораторные работы
\begin{enumerate}
\item компьютерный класс.
\end{enumerate}
\item Прочее
\begin{enumerate}
\item рабочее место преподавателя, оснащенное компьютером с доступом в Интернет, \item рабочие места студентов, оснащенные компьютерами с доступом в Интернет, предназначенные для работы в электронной образовательной среде.
\end{enumerate}
\end{enumerate}

\newpage
\begin{flushright}
\textbf{Приложение 1\\
к рабочей программе дисциплины\\
«Функциональное программирование»}
\end{flushright}
\section*{Аннотация рабочей программы}

Дисциплина «Функциональное программирование» является частью базовой части профессионального цикла дисциплин подготовки студентов по направлению подготовки Прикладная математика и информатика.
Дисциплина реализуется на факультете ИТиП НИУ ИТМО кафедрой КТ.

Дисциплина нацелена на формирование общекультурных компетенций: ОК.14, ОК.15, ОК.16 и профессиональных компетенций: ПК.3, ПК.4, ПК.5, ПК.7 выпускника. Дисциплина относится к циклу дисциплин специальной профессиональной подготовки. Изучение дисциплины основывается на знаниях и умениях, получаемых студентами при изучении дисциплин "Дискретная математика", "Основы программирования". Знания и умения, полученные при изучении дисциплины, используются при защитах курсовых и выпускной квалификационной работ, прохождении практик.

Преподавание дисциплины предусматривает следующие формы организации учебного процесса: лекции, лабораторные работы, самостоятельная работа студента и консультации.

Программой дисциплины предусмотрены следующие виды контроля: текущий контроль успеваемости в форме
проверки выполнения лабораторных работ, защиты лабораторных работ (тестирований), рубежный контроль в форме защит лабораторных работ (тестирований) и промежуточный контроль в форме экзамена.

Общая трудоемкость освоения дисциплины составляет 4 зачетных единиц, 136 часов. Программой дисциплины предусмотрены: лекционные   (34 часов), лабораторные (34 часов) работы и самостоятельная работа студента (68 часов).

\newpage
\begin{flushright}
\textbf{Приложение 2\\
к рабочей программе дисциплины\\
«Функциональное программирование»}
\end{flushright}

\section*{ТЕХНОЛОГИИ И ФОРМЫ ПРЕПОДАВАНИЯ\\
Рекомендации по организации и технологиям обучения для преподавателей}

\def\thesubsection{\Roman{subsection}}

\subsection{Образовательные технологии}

{\parindent0pt
Преподавание дисциплины ведется с применением следующих видов образовательных технологий:\\
\textbf{Информационные технологии}: использование электронных образовательных ресурсов при подготовке к лекциям и лабораторным работам разделов 1 и 2.

%TODO: сделать работу в команде и прочую муть
% 3. Case-study, 4. Игра, 5. Проблемное обучение, 6. Контекстное обучение, 7. Обучение на основе опыта, 8. Индивидуальное обучение, 9. Междисциплинарное обучение, 10. Опережающая самостоятельная работа
}

\subsection{Виды и содержание учебных занятий}

\subsubsection{Раздел 1. «Понятие о функциональном программировании»}

{\parindent0pt
\setdescription{leftmargin=\parindent,labelindent=1cm}
\setitemize[1]{leftmargin=1.5cm}

\textbf{Теоретические занятия (лекции)~— 18 часов.}
\begin{description}
\item[Лекция 1.] «Функциональное программирование». Информационная лекция. Рассматриваются следующие вопросы: \begin{itemize}
\item Императивные и функциональные языки\item Функциональный стиль
\end{itemize}\item[Лекция 2.] «Введение в язык Haskell». Информационная лекция. Рассматриваются следующие вопросы: \begin{itemize}
\item История\item Типы данных\item Конструкторы\item Функции\item Рекурсия
\end{itemize}\item[Лекция 3.] «Функции высших порядков». Информационная лекция. Рассматриваются следующие вопросы: \begin{itemize}
\item Определение\item Примеры
\end{itemize}\item[Лекция 4.] «Лямбда-исчисление». Информационная лекция. Рассматриваются следующие вопросы: \begin{itemize}
\item Основы лямбда-исчисления\item Рекурсия в лямбде\item Чистое лямбда-исчисление
\end{itemize}
\end{description}




\textbf{Лабораторный практикум~— 18 часов, 4 работ.}
\begin{description}
\item[Лабораторная работа 1.] «Функциональное программирование: проба пера». Выполняется индивидуально в лаборатории «Компьютерный класс». Цели работы: \begin{itemize}
\item Студенты должны Написать программу для вычисления приближенного значения числа e по формуле для разложения e\^x в ряд Тейлора.\item Предложить программы на языке Паскаль, написанные в традиционном императивном и функциональном стилях.\item Студенты должны понимать особенности функционального стиля программирования\item Студенты должны уметь писать программы в функциональном стиле на традиционных языках программирования
\end{itemize}\item[Лабораторная работа 2.] «Рекурсия». Выполняется индивидуально в лаборатории «Компьютерный класс». Цели работы: \begin{itemize}
\item Написать функцию, определяющую количество строк в списке, содержащих хотя бы одну букву (буквой будем называть символ, для которого заданнаяк функция выдает значение True).\item Студенты должны знать основные конструкции языка Haskell\item Студенты должны уметь писать программы на Haskell небольшого размера
\end{itemize}\item[Лабораторная работа 3.] «Работа со строками». Выполняется индивидуально в лаборатории «Компьютерный класс». Цели работы: \begin{itemize}
\item Написать функцию, вычисляющую длину самой длинной строки в заданном списке строк.\item Студенты должны быть знакомы с функциями высших порядков\item Студенты должны уметь реализовывать функции, принимаюищие другие функции в качестве аргумента\item Студенты должны уметь применять на практике функции высших порядков, такие как свертка и map
\end{itemize}\item[Лабораторная работа 4.] «Дерево». Выполняется индивидуально в лаборатории «Компьютерный класс». Цели работы: \begin{itemize}
\item Дерево задано с помощью следующего описания структуры данных. data Tree a = Node a [Tree a]\item то есть дерево представляет собой корневой узел, содержащий некоторое значение произвольного типа a и список поддеревьев. Написать функцию, вычисляющую высоту дерева.\item Студенты должны понимать основы лямбда-исчисления\item Студенты должны уметь строить несложные интерфейсы\item Студенты должны уметь писать программы, состоящие из нескольких модулей, связанных интерфейсами
\end{itemize}
\end{description}

\textbf{Управление самостоятельной работой студента.}
\begin{itemize}
\item Консультации по выполнению лабораторных работ.
\end{itemize}
}


\subsubsection{Раздел 2. «Интерпретация и компиляция функциональных программ»}

{\parindent0pt
\setdescription{leftmargin=\parindent,labelindent=1cm}
\setitemize[1]{leftmargin=1.5cm}

\textbf{Теоретические занятия (лекции)~— 16 часов.}
\begin{description}
\item[Лекция 5.] «Представление функциональных программ». Информационная лекция. Рассматриваются следующие вопросы: \begin{itemize}
\item Компиляция с языка Haskell в расширенное лямбда-исчисление\item Компиляция case-выражения\item Компиляция сопоставления с образцом и связывания\item Представление программ расширенного лямбда-исчисления
\end{itemize}\item[Лекция 6.] «Интерпретация». Информационная лекция. Рассматриваются следующие вопросы: \begin{itemize}
\item Eval/Apply интерпретатор\item Функциональная SECD-машина\item Функциональные эквиваленты императивных программ
\end{itemize}
\end{description}




\textbf{Лабораторный практикум~— 16 часов, 2 работ.}
\begin{description}
\item[Лабораторная работа 5.] «Граф». Выполняется индивидуально в лаборатории «Компьютерный класс». Цели работы: \begin{itemize}
\item Структура графа задана списками смежности номеров вершин, то есть списком, элементами которого являются пары, состоящие из номера вершины и списка вершин, инцидентных ей\item Написать функцию, которая проверяет, существует ли в графе путь, соединяющий вершины с двумя заданными номерами.
\end{itemize}\item[Лабораторная работа 6.] «Лямбда-исчисление». Выполняется индивидуально в лаборатории «Компьютерный класс». Цели работы: \begin{itemize}
\item Выполнить редукцию выражения. В нормальном и аппликативном порядке редукций. В обоих случаях найти нормальную форму (НФ) и слабую заголовочную нормальную форму (СЗНФ) выражения.\item Студенты должны понимать принципы компиляции и интерпретации программ\item Студенты должны уметь выполнять редукцию лямбда-выражений\item Студенты должны уметь реализовывать структуры данных на функциональных языках программирования\item в рамках которого студенты отвечают на теоретические вопросы, проектируют и реализуют программы с применением навыков и умений,\item полученных в процессе освоения дисциплины.
\end{itemize}
\end{description}

\textbf{Управление самостоятельной работой студента.}
\begin{itemize}
\item Консультации по выполнению лабораторных работ.
\end{itemize}
}



\subsubsection{Курсовые работы}
{\parindent0pt
\setdescription{leftmargin=\parindent,labelindent=0cm}
Не предусмотрены.
}

\newpage
\pagestyle{empty}
\begin{landscape}
\begin{flushright}
\textbf{Приложение 3\\
к рабочей программе дисциплины\\
«Функциональное программирование»}
\end{flushright}

Трудоемкость освоения дисциплины составляет 136 часов, из них 68 часов аудиторных занятий и 68 часов, отведенных на самостоятельную работу студента.
Рекомендации по распределению учебного времени по видам самостоятельной работы и разделам дисциплины приведены в таблице.
Формы контроля и критерии оценивания приведены в Приложениях 4 и 5 к Рабочей программе.

\begin{center}
\begin{longtable}{|p{0.3\textwidth}|p{0.6\textwidth}|c|p{0.3\textwidth}|}\hline
\multicolumn{1}{|c|}{\bfseries \pb{~\\Вид работы\\~}} &
\multicolumn{1}{c|}{\bfseries Содержание (перечень вопросов)} &
\bfseries Трудоемкость, час. &
\multicolumn{1}{c|}{\bfseries Рекомендации}\\\hline
\multicolumn{4}{|>{\columncolor[gray]{.9}}c|}{\bfseries Раздел 1. «Понятие о функциональном программировании»}\\\hline
Повторение материала лекции №1 & \begin{itemize}
\item Императивные и функциональные языки\item Функциональный стиль
\end{itemize} & 4 & См. содержимое лекции\\\hline
Выполнение лабораторной работы №1 & \begin{itemize}
\item Студенты должны Написать программу для вычисления приближенного значения числа e по формуле для разложения e\^x в ряд Тейлора.\item Предложить программы на языке Паскаль, написанные в традиционном императивном и функциональном стилях.\item Студенты должны понимать особенности функционального стиля программирования\item Студенты должны уметь писать программы в функциональном стиле на традиционных языках программирования
\end{itemize} & 4 & См. задание на лабораторную работу\\\hline
Повторение материала лекции №2 & \begin{itemize}
\item История\item Типы данных\item Конструкторы\item Функции\item Рекурсия
\end{itemize} & 4 & См. содержимое лекции\\\hline
Выполнение лабораторной работы №2 & \begin{itemize}
\item Написать функцию, определяющую количество строк в списке, содержащих хотя бы одну букву (буквой будем называть символ, для которого заданнаяк функция выдает значение True).\item Студенты должны знать основные конструкции языка Haskell\item Студенты должны уметь писать программы на Haskell небольшого размера
\end{itemize} & 4 & См. задание на лабораторную работу\\\hline
Повторение материала лекции №3 & \begin{itemize}
\item Определение\item Примеры
\end{itemize} & 5 & См. содержимое лекции\\\hline
Выполнение лабораторной работы №3 & \begin{itemize}
\item Написать функцию, вычисляющую длину самой длинной строки в заданном списке строк.\item Студенты должны быть знакомы с функциями высших порядков\item Студенты должны уметь реализовывать функции, принимаюищие другие функции в качестве аргумента\item Студенты должны уметь применять на практике функции высших порядков, такие как свертка и map
\end{itemize} & 5 & См. задание на лабораторную работу\\\hline
Повторение материала лекции №4 & \begin{itemize}
\item Основы лямбда-исчисления\item Рекурсия в лямбде\item Чистое лямбда-исчисление
\end{itemize} & 5 & См. содержимое лекции\\\hline
Выполнение лабораторной работы №4 & \begin{itemize}
\item Дерево задано с помощью следующего описания структуры данных. data Tree a = Node a [Tree a]\item то есть дерево представляет собой корневой узел, содержащий некоторое значение произвольного типа a и список поддеревьев. Написать функцию, вычисляющую высоту дерева.\item Студенты должны понимать основы лямбда-исчисления\item Студенты должны уметь строить несложные интерфейсы\item Студенты должны уметь писать программы, состоящие из нескольких модулей, связанных интерфейсами
\end{itemize} & 5 & См. задание на лабораторную работу\\\hline
\multicolumn{1}{|r|}{Итого по разделу 1} &  & 36 & \\\hline
\multicolumn{4}{|>{\columncolor[gray]{.9}}c|}{\bfseries Раздел 2. «Интерпретация и компиляция функциональных программ»}\\\hline
Повторение материала лекции №5 & \begin{itemize}
\item Компиляция с языка Haskell в расширенное лямбда-исчисление\item Компиляция case-выражения\item Компиляция сопоставления с образцом и связывания\item Представление программ расширенного лямбда-исчисления
\end{itemize} & 8 & См. содержимое лекции\\\hline
Повторение материала лекции №6 & \begin{itemize}
\item Eval/Apply интерпретатор\item Функциональная SECD-машина\item Функциональные эквиваленты императивных программ
\end{itemize} & 8 & См. содержимое лекции\\\hline
Выполнение лабораторной работы №5 & \begin{itemize}
\item Структура графа задана списками смежности номеров вершин, то есть списком, элементами которого являются пары, состоящие из номера вершины и списка вершин, инцидентных ей\item Написать функцию, которая проверяет, существует ли в графе путь, соединяющий вершины с двумя заданными номерами.
\end{itemize} & 8 & См. задание на лабораторную работу\\\hline
Выполнение лабораторной работы №6 & \begin{itemize}
\item Выполнить редукцию выражения. В нормальном и аппликативном порядке редукций. В обоих случаях найти нормальную форму (НФ) и слабую заголовочную нормальную форму (СЗНФ) выражения.\item Студенты должны понимать принципы компиляции и интерпретации программ\item Студенты должны уметь выполнять редукцию лямбда-выражений\item Студенты должны уметь реализовывать структуры данных на функциональных языках программирования\item в рамках которого студенты отвечают на теоретические вопросы, проектируют и реализуют программы с применением навыков и умений,\item полученных в процессе освоения дисциплины.
\end{itemize} & 8 & См. задание на лабораторную работу\\\hline
\multicolumn{1}{|r|}{Итого по разделу 2} &  & 32 & \\\hline

\end{longtable}
\end{center}

\end{landscape}


\newpage
\pagestyle{plain}
\begin{flushright}
\textbf{Приложение 4\\
к рабочей программе дисциплины\\
«Функциональное программирование»}
\end{flushright}

\section*{ОЦЕНОЧНЫЕ СРЕДСТВА И МЕТОДИКИ ИХ ПРИМЕНЕНИЯ}

Оценивание уровня учебных достижений студента осуществляется в виде текущего контроля и промежуточной аттестации в соответствии с Положением о проведении текущего контроля успеваемости и промежуточной аттестации студентов НИУ ИТМО.

\subsection*{Фонды оценочных средств}

Фонды оценочных средств, позволяющие оценить РО по данной дисциплине, включают в себя:
\begin{itemize}
\item шаблоны отчётов по лабораторным работам, выдаются индивидуально.
\end{itemize}

\subsection*{Критерии оценивания}




\noindent\textbf{Лабораторные работы}\\
\textit{Допуск за защите ЛР}\\
Допуск к защите ЛР происходит в форме устного тестирования направленного на проверку самостоятельности выполнения ЛР. При ответе на более чем 60\% вопросов студент допускается к защите ЛР.\\

\noindent\textit{Защита ЛР}\\
Отчет по лабораторной работе представляется в электронном виде в формате, предусмотренном шаблоном отчета по лабораторной работе.
Защита отчета проходит в форме доклада студента по выполненной работе и ответов на вопросы преподавателя.
В случае если содержание и оформление отчета, а также поведение студента во время защиты соответствуют указанным требованиям, студент получает максимальное количество баллов.

Основаниями для снижения количества баллов в диапазоне от max до min являются:
\begin{itemize}
\item небрежное выполнение,
\item низкое качество графического материала.
\end{itemize}

Отчет не может быть принят и подлежит доработке в случае:
\begin{itemize}
\item отсутствия необходимых разделов,
\item отсутствия необходимого графического материала,
\item некорректной обработки результатов.
\end{itemize}




\newpage
\begin{landscape}
\begin{flushright}
\textbf{Приложение 5\\
к рабочей программе дисциплины\\
«Функциональное программирование»}
\end{flushright}


\section*{\Large Таблица планирования результатов обучения студентов 4 курса по дисциплине «Функциональное программирование» в 7 семестре}

\begin{adjustwidth}{ -0.5cm}{ -0.5cm}\begin{center}
\begin{tabular}{|c|  c|c| c|c| c|c| c|c| c|c| c|c| c|c| c|c| c|c| c|c|   c|c|}\hline
\multicolumn{1}{|c|}{\multirow{4}{*}{\pb{\bfseries~\\Формы\\контроля\\~}}} &
\multicolumn{10}{c|}{\pb{\bfseries~\\Модуль~13\\~}} & \multicolumn{10}{c|}{\pb{\bfseries~\\Модуль~14\\~}} &  
\multicolumn{2}{c|}{\multirow{3}{*}{\pb{\bfseries~\\Промежу-\\точная\\аттестация\\~}}}\\\cline{2-21}

&

\multicolumn{8}{c|}{\pb{\bfseries Текущий контроль по точкам}} &
\multicolumn{2}{c|}{\multirow{2}{*}{\pb{\bfseries Рубежный}}} &

\multicolumn{8}{c|}{\pb{\bfseries Текущий контроль по точкам}} &
\multicolumn{2}{c|}{\multirow{2}{*}{\pb{\bfseries Рубежный}}} &

\multicolumn{2}{c|}{~}\\\cline{2-9}\cline{12-19}

&

\multicolumn{2}{c|}{\pb{\bfseries 1}} &
\multicolumn{2}{c|}{\pb{\bfseries 2}} &
\multicolumn{2}{c|}{\pb{\bfseries 3}} &
\multicolumn{2}{c|}{\pb{\bfseries 4}} &
\multicolumn{2}{c|}{~} &

\multicolumn{2}{c|}{\pb{\bfseries 1}} &
\multicolumn{2}{c|}{\pb{\bfseries 2}} &
\multicolumn{2}{c|}{\pb{\bfseries 3}} &
\multicolumn{2}{c|}{\pb{\bfseries 4}} &
\multicolumn{2}{c|}{~} &

\multicolumn{2}{c|}{~} \\\cline{2-23}

&

\pb{\tiny min} &
\pb{\tiny max} &
\pb{\tiny min} &
\pb{\tiny max} &
\pb{\tiny min} &
\pb{\tiny max} &
\pb{\tiny min} &
\pb{\tiny max} &
\pb{\tiny min} &
\pb{\tiny max} &

\pb{\tiny min} &
\pb{\tiny max} &
\pb{\tiny min} &
\pb{\tiny max} &
\pb{\tiny min} &
\pb{\tiny max} &
\pb{\tiny min} &
\pb{\tiny max} &
\pb{\tiny min} &
\pb{\tiny max} &


\pb{\tiny min} &
\pb{\tiny max} \\\hline

\pb{Лабораторная работа} & 3 & 6 & 4 & 7 & 4 & 7 & 5 & 7 &  &  &  &  & 6 & 11 &  &  & 8 & 12 &  &  &  & \\\hline
\pb{Рубежное тестирование} &  &  &  &  &  &  &  &  & 6 & 10 &  &  &  &  &  &  &  &  & 6 & 10 &  & \\\hline
\pb{Личностные качества} &  &  &  &  &  &  & 3 & 5 &  &  &  &  &  &  &  &  & 3 & 5 &  &  &  & \\\hline
\pb{Экзамен} &  &  &  &  &  &  &  &  &  &  &  &  &  &  &  &  &  &  &  &  & 12 & 20\\\hline
\pb{Балловая стоимость\\одной точки} & 3 & 6 & 4 & 7 & 4 & 7 & 8 & 12 & 6 & 10 & 0 & 0 & 6 & 11 & 0 & 0 & 11 & 17 & 6 & 10 & 12 & 20\\\hline
\pb{Накопление баллов} & 3 & 6 & 7 & 13 & 11 & 20 & 19 & 32 & 25 & 42 & 0 & 0 & 6 & 11 & 6 & 11 & 17 & 28 & 23 & 38 &  & \\\hline

\multicolumn{21}{|r|}{\pb{\bfseries Итого:}} &60 &  100\\\hline
\end{tabular}
\end{center}\end{adjustwidth}



\end{landscape}

