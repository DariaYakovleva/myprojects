% !TEX encoding = UTF-8
% !TEX program = pdflatex
\documentclass[12pt]{article}

\usepackage[T2A]{fontenc}
\usepackage[utf8]{inputenc}
\usepackage[russian]{babel}
\textheight=24cm
\textwidth=16cm
\oddsidemargin=0pt
\topmargin=-1.5cm

\title{Лабораторная работа №2}
\author{\copyright~~Дарья Яковлева}
\date{31 марта 2016 года}
\begin{document}
\maketitle
\thispagestyle{empty}
Вариант 8. Визуализация переменных в Си

Описания переменных в Си. Сначала следует имя типа, затем
разделенные  запятой  имена  переменных.  Переменная  может  быть
указателем, в этом случае перед ней идет звездочка (возможны и
указатели на указатели, и т. д.). Описаний может быть несколько.
Используйте один терминал для всех имен переменных и имен типов.

Пример

	\begin{verbatim}
		int a, *b, ***c, d;
	\end{verbatim}

\begin{itemize}
	\item \textbf{Разработка грамматики}

			Построим грамматику.

			S $\rightarrow$ D

			D $\rightarrow$ TAD

			D $\rightarrow$ $\varepsilon$

			A $\rightarrow$ ' 'B

			B $\rightarrow$ VC

			C $\rightarrow$ ,VC

			C $\rightarrow$ ;

			V $\rightarrow$ *V

			V $\rightarrow$ a

			T $\rightarrow$ a

			S -- стартовый нетерминал

			D -- одно описание переменных

			T -- тип переменных

			A -- переход от типа к переменным

			В -- первая переменная

			С -- вторая и все остальные переменные

			V -- переменная

			Левой рекурсии нет.

	\item \textbf{Построение лексического анализатора}
		\begin{verbatim}
			enum Token {
			    COMMA, SEMICOLON, STAR, NAME, SPACE, END;
			}
			
			public class LexicalAnalyzer {
			
			    String input;
			    int curPos;
			    Token curToken;
			    char curChar;
			    int haveType;
			
			    public LexicalAnalyzer(String input) {
			        this.input = input + "$";
			        curPos = 0;
			        nextChar();
			        haveType = 0;
			    }
			
			    void nextChar() {
			        curChar = input.charAt(curPos);
			        curPos++;
			    }
			
			    public Token curToken() {
			        return curToken;
			    }
			
			    boolean isBlank(char c) {
			        return c == ' ' || c == '\r' || c == '\n' || c == '\t';
			    }
			
			    boolean next() {
			        return curPos < input.length();
			    }
			
			    public void nextToken() throws ParseException {
			        while (next() && isBlank(curChar)) {
			            nextChar();
			        }
			        if (isBlank(curChar)) {
			            throw new ParseException("End of input string at position ", curPos);
			        }
			        if (haveType == 1) {
			            curToken = Token.SPACE;
			            haveType = 2;
			            return;
			        }
			        if (curChar == '*') {
			            curToken = Token.STAR;
			            nextChar();
			        } else if (curChar == ';') {
			            curToken = Token.SEMICOLON;
			            nextChar();
			            haveType = 0;
			        } else if (curChar == ',') {
			            curToken = Token.COMMA;
			            nextChar();
			        } else if (curChar == '$') {
			            curToken = Token.END;
			        } else {
			            String type = "" + curChar;
			            while (next() && !isBlank(curChar) && curChar != ',' && curChar != ';') {
			                nextChar();
			                type += curChar;
			            }
			            curToken = Token.NAME;
			            if (haveType == 0) {
			                haveType = 1;
			            }
			        }
			    }
			}		
		\end{verbatim}

	\item \textbf{Построение синтаксического анализатора}

		\begin{tabular} {|c|c|c|}
			\hline
				\textbf{Нетерминал} & \textbf{FIRST} & \textbf{FOLLOW} \\
			\hline
				S & $\varepsilon$,  a & \$ \\
			\hline
				D & $\varepsilon$, a & \$ \\
			\hline
				A & space & a \\
			\hline
				B & *, a & a \\
			\hline
				C & \textbf{,}, ; & a \\
			\hline
				V & *, a & \textbf{,}, ; \\
			\hline
				T & a & ' ' \\
			\hline
		\end{tabular}


		Структура данных для хранения дерева
		\begin{verbatim}
			class Tree {
			    String node;
			    List<Tree> children;
			    public Tree(String node, Tree... children) {
			        this.node = node;
			        this.children = Arrays.asList(children);
			    }
			    public Tree(String node) {
			        this.node = node;
			        this.children = new ArrayList<>();
			    }
			}
		\end{verbatim}

		Синтаксический анализатор

		\begin{verbatim}
			public class Parser {
			    LexicalAnalyzer lex;
			
			    Tree parse(String input) throws ParseException {
			        lex = new LexicalAnalyzer(input);
			        lex.nextToken();
			        return S();
			    }
			
			    Tree S() throws ParseException {
			        switch (lex.curToken()) {
			            case NAME:
			                //D
			                Tree sub = D();
			                return new Tree("S", sub);
			            case END:
			                //eps
			                return new Tree("S");
			            default:
			                throw new AssertionError();
			        }
			    }
			    Tree D() throws ParseException {
			        switch (lex.curToken()) {
			            case NAME:
			                //T
			                Tree sub = T();
			                //A
			                Tree cont = A();
			                //D
			                Tree cont2 = D();
			                return new Tree("D", sub, cont, cont2);
			            case END:
			                //eps
			                return new Tree("D", new Tree("$"));
			            default:
			                throw new AssertionError();
			        }
			    }
			    Tree T() throws ParseException {
			        switch (lex.curToken()) {
			            case NAME:
			                lex.nextToken();
			                return new Tree("T", new Tree("a"));
			            default:
			                throw new AssertionError();
			        }
			    }
			    Tree A() throws ParseException {
			        switch (lex.curToken()) {
			            case SPACE:
			                lex.nextToken();
			                Tree cont = B();
			                return new Tree("A", new Tree("space"), cont);
			            default:
			                throw new AssertionError();
			        }
			    }
			    Tree B() throws ParseException {
			        switch (lex.curToken()) {
			            case STAR:
			                Tree sub = V();
			                Tree cont = C();
			                return new Tree("B", sub, cont);
			            case NAME:
			                sub = V();
			                cont = C();
			                return new Tree("B", sub, cont);
			            default:
			                throw new AssertionError();
			        }
			    }
			    Tree V() throws ParseException {
			        switch (lex.curToken()) {
			            case STAR:
			                lex.nextToken();
			                Tree cont = V();
			                return new Tree("V", new Tree("*"), cont);
			            case NAME:
			                lex.nextToken();
			                return new Tree("V", new Tree("a"));
			            default:
			                throw new AssertionError();
			        }
			    }
			
			    Tree C() throws ParseException {
			        switch (lex.curToken()) {
			            case COMMA:
			                lex.nextToken();
			                Tree cont = V();
			                Tree cont2 = C();
			                return new Tree("C", new Tree(","), cont, cont2);
			            case SEMICOLON:
			                lex.nextToken();
			                return new Tree("C", new Tree(";"));
			            default:
			                throw new AssertionError();
			        }
			    }
			}
		\end{verbatim}				

	\item \textbf{Визуализация дерева разбора}

		Тест
		\begin{verbatim}				
			int a, **b; double c; 
		\end{verbatim}				

		Дерево разбора
		\begin{verbatim}				
			S -> D -> T -> a
		     | -> A -> space
		     |    | -> B -> V -> a
		     |         | -> C -> ,
		     |              | -> V -> *
		     |              |    | -> V -> *
		     |              |         | -> V -> a
		     |              | -> C -> ;
		     | -> D -> T -> a
		          | -> A -> space
		          |    | -> B -> V -> a
		          |         | -> C -> ;
		          | -> D -> $
		\end{verbatim}				

	\item \textbf{Подготовка набора тестов}

		\begin{tabular} {|l|l|}
			\hline
				\textbf{Тест} & \textbf{Описание}\\
			\hline
				int a; & Простой тест \\
			\hline
				int *a; & Простой тест №2 \\
			\hline
				double a, b, c; & Тест на правило C $\rightarrow$ ,VC \\
			\hline
				int *a, **b; & Тест на правило V $\rightarrow$ *V \\
			\hline
				int a; double c; & Тест на правило D $\rightarrow$ TAD \\
			\hline
				  & Тест на правило D $\rightarrow$ $\varepsilon$ \\
			\hline
				int a; float *s, *****d, e; & Случайный тест \\
			\hline
		\end{tabular}


\end{itemize}

\end{document}

%%% Local Variables:
%%% fill-column: 10
%%% mode: latex
%%% coding: utf-8
%%% TeX-PDF-mode: t
%%% End: