\begin{problem}{Два подарка}{стандартный ввод}{стандартный вывод}{2 секунды}{256 мегабайт}

Сеня выбирает себе подарки на новый год. Он знает, что Дед Мороз купит ему ровно два подарка: один
якобы от мамы, а другой якобы от папы.

В магазине, где Дед Мороз будет покупать подарки,
продаётся $n$ подарков, про каждый подарок известна его цена: цена $i$-го подарка равна $a_i$ рублей. 
Сеня знает, что Дед Мороз может потратить на покупку его подарков не больше $x$ рублей.
Разумеется, он хочет получить как можно более дорогие подарки. Таким образом, он хочет выбрать два
различных подарка с максимальной суммарной ценой, но при этом она не должна превышать $x$.

Помогите Сене выбрать себе подарки.

\InputFile
Первая строка ввода содержит два целых числа: $n$ и $x$ ($2 \le n \le 100\,000$, $2 \le x \le 10^9$).
Вторая строка ввода содержит $n$ целых чисел: $a_1, a_2, \ldots, a_n$ ($1 \le a_i \le 10^9$).
Гарантируется, что существует два подарка с суммарной ценой не больше $x$.

\OutputFile
Выведите одно целое число: максимальную суммарную цену двух различных подарков, не превышающую $x$.

\Example

\begin{example}%
\exmp{
6 18
5 3 10 2 4 9
}{
15
}%
\end{example}

\end{problem}
