\begin{problem}{Реформы в королевстве}{стандартный ввод}{стандартный вывод}{2 секунды}{256 мегабайт}

В одном королевстве есть $n$ городов, расположенных вдоль длинной прямой дороги, $i$-й город
расположен на расстоянии $x_i$ километров от начала дороги ($0 \le x_1 < x_2 < \ldots < x_n \le 10^9$). 

В ближайшее время король планирует
провести реформу управления королевством и разделить его на $k$ провинций. 
Каждый город должен войти ровно в одну провинцию.

В каждую провинцию войдет от $a$ до $b$ городов, причем эти города должны иметь следующие 
подряд номера. Таким образом, каждая провинция характеризуется числами $i$ и $l$, 
для которых $1 \le i$, $i + l - 1 \le n$, $a \le l \le b$ и в провинцию входят
города с номерами $i, i + 1, \ldots, i + l - 1$. 

Чтобы минимизировать затраты на обслуживание провинций, король хочет, чтобы максимальное расстояние
между городами, входящими в одну провинцию, было как можно меньше. Помогите королю выполнить
разделение королевства.


\InputFile
Первая строка ввода содержит четыре целых числа: $n$, $k$, $a$ и $b$ ($1 \le n \le 200$, $1 \le k \le n$,
$1 \le a \le b \le n$, $ak \le n \le bk$).
Вторая строка ввода содержит $n$ целых чисел: $x_1, x_2, \ldots, x_n$ ($0 \le x_1 < x_2 < \ldots < x_n \le 10^9$). 

\OutputFile
Выведите одно целое число: минимальное возможное $z$, такое чтобы можно было разбить города на провинции
описанным образом, и расстояние между городами внутри одной провинции не превышало $z$.

\Example

\begin{example}%
\exmp{
6 2 2 4
1 2 3 4 6 13
}{
7
}%
\end{example}

\Explanation

В примере оптимально первые 4 города объединить в первую провинцию, а пятый и шестой --- во вторую.
Максимальное расстояние между двумя городами в одной провинции: $13 - 6 = 7$.

\end{problem}
