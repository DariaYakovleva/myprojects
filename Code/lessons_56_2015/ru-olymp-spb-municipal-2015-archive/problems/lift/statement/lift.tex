\begin{problem}{Лифт}{стандартный ввод}{стандартный вывод}{2 секунды}{256 мегабайт}

Миша живет на $n$-м этаже. Когда Миша заходит в подъезд, он смотрит, на каком этаже 
в этот момент находится лифт и решает, вызвать лифт или пойти по лестнице. 

Сегодня лифт находится на $k$-м этаже.
Миша заходит в подъезд на 1 этаже.
Он поднимается на один этаж за $a$ секунд. Лифт перемещается на один этаж за $b$
секунд. Временем входа в лифт и выхода из лифта, а также перемещения к лестнице и обратно
можно пренебречь.

Помогите Мише принять решение, выведите, за какое время он попадет на свой этаж
на лифте и по лестнице, соответственно.

\InputFile
На ввод подаются целые числа: $n$, $k$, $a$ и $b$.

$2 \le n \le 100$, $1 \le k \le 100$, $1 \le a, b \le 1000$.

\OutputFile
Выведите два целых числа: время, за которое Миша поднимется на свой этаж на лифте,
и время, за которое Миша поднимется на свой этаж по лестнице.

\Example

\begin{example}%
\exmp{
15 8 5 3
}{
63 70
}%
\end{example}

\Explanation

В примере лифту необходимо $7\times 3=21$ секунда, чтобы спуститься с 8 этажа
и затем $14\times 3=42$ секунды, чтобы подняться на 15 этаж, где живет Миша.
Мише же необходимо $14\times 5 = 70$ секунд, чтобы подняться на 15 этаж по лестнице.

\end{problem}
